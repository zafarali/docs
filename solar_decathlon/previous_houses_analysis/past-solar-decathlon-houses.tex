\input{sleek-journal-article.tex}

\title{Past Solar Decathlon Houses}
\author{
\normalsize
{ Calem Bendell } \\
Hyvedev, McGill University \\
\href{mailto:calem.bendell@mail.mcgill.ca}{calem.bendell@mail.mcgill.ca}
}
\date{}

\begin{document}
\twocolumn[\maketitle
{\large Version 0.1} \\ 
This is the first and potentially the last edition of this paper.
Expect no updates unless requested.
\begin{center}
	\includegraphics{gfx/hyvedev-logo.pdf}
\end{center}]

\section{Introduction}

\section{Primary Points}

\section{Evidence}

\section{Notes}

	\subsection{2012 Competition}
	
		\subsubsection{Luuku House (Helsinki)}
			
			Used automation for night cooling, but did not particularly specify details.
			Which parts were automated are doubtful, since many specifications were given for the limits of the system, but some of the parameters, such as recirculation air flow rate, are given as constants.
			Altogether, their report is woefully unclear for automation.
			
		\subsubsection{Wuppertal (Wuppertal, Germany)}
		
			Combined a "compact unit" for HVAC/DHW.
			This unit provided heating, cooling, ventilation, and hot water with "all the control needed."
			No processing unit is specified, so this must have been controlled by a reactionary microprocessor.
			
			
		\subsubsection{SML House (Valencia)}
		
			Featured a wireless automation system and suggested that was special.
			Basic control of the house including lighting and temperature seems to have been provided through a phone flash app.
			Unable to locate a copy of this flash app.
			Given that the app is in flash, a generally inadvisable platform, and no processing units are suggested, there seems to be no automatic control.
			
		\subsubsection{LOW$_3$}
		
			Control system takes a significant amount of energy, but no higher level control system is specified.
			
		\subsubsection{title}

\section{Conclusions}

\end{document}